Das Durchführen dieses Versuches hat gezeigt, dass unzählige bösartige Webcrawler im Internet ihr Unwesen treiben. Die Zahl von ihnen wird täglich größer. Das Betreiben einer Tarpit als Schutzmaßnahme vor Webcrawlern und ihren Angriffen ist jedoch nicht rentabel. Webcrawler verändern sich über die Zeit, werden schlauer, erkennen Gefahren und reagieren darauf. Um mit den Webcrawlern Schritt halten zu können müssen sich auch die Tarpits dementsprechend weiterentwickeln aber sie hinken dabei den Webcrawlern immer einen Schritt hinterher: Auf ein Angriffsmuster kann erst reagiert werden, nachdem es schon einmal angewendet wurde, doch dann ist es meist schon zu spät.\\
Das Betreiben und Weiterentwickeln von Tarpits ist ein ressourcen- und zeitaufwendiges Unterfangen, dessen Kosten ein Unternehmen erst einmal stemmen muss. Das investierte Geld wäre in qualitative Antivirensoftware, Spamfilter oder dergleichen besser investiert. Auch kann eine einzige Tarpit niemals gegen die schier endlose Anzahl an Webcrawlern aufkommen. Fängt man einen, können auf dem Webserver des Betreibers des Webcrawlers noch zehn weitere auf ihren Einsatz warten. Ein Verbund von Tarpits wäre erforderlich, dessen Koordination erneut Unsummen verschlingen würde. Des Weiteren ist zu bedenken, dass sich auch die Betreiber von Webcrawlern zusammenschließen könnten und ein gezielter Angriff von Webcrawlern, welche über ein oder gar mehrere Botnetze betrieben werden, nahezu jeden Webserver in die Knie zwingen könnte.\\
So interessant die Vorstellung einer Tarpit und dessen Konzept der \glqq Rache\grqq\space an einem Webcrawler auch klingt, ist die Umsetzung zum jetzigen Zeitpunkt wirtschaftlich nicht rentabel und praktisch auch nahezu unmöglich. Hier heißt es einfach abwarten was die Zukunft bringt. Doch spätestens dann, wenn die rapiden Fortschritte in der künstlichen Intelligenz Früchte tragen oder die Theorien des Quantencomputers zur Realität werden, müssen wir uns alle Gedanken über unsere Sicherheit im Internet machen, denn unsere jetzigen Sicherheitskonzepte werden bei weitem nicht mehr ausreichend sein.