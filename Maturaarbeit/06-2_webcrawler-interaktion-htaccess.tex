\emph{.htaccess} ist der Name einer Textdatei, die für die Konfiguration eines Apache Webservers verwendet wird. Da der Webserver von Apache die im Internet mit einem Marktanteil von 46,9\%\cite[Stand 24.03.2018:][]{statistik-webserver} am weitesten verbreitete Serversoftware ist, ist auch die Verbreitung von .htaccess dementsprechend hoch.
%TODO LÖSCHEN
%Die .htaccess-Datei kann dabei an jeden x-beliebigen Ort innerhalb des Webservers platziert werden. Die Regeln sind dann für diesen Ordner und alle Unterordner wirksam\cite{htaccess}.
Apache akzeptiert hierbei eine Vielzahl an Regeln in der .htaccess-Datei; darunter auch Regeln, welche Anfragen anhand der IP-Adresse und/oder des User Agent\footnote{Ein \emph{User Agent} ist eine Zeichenfolge, welche bei jedem Aufruf einer Seite an den Webserver übermittelt wird. In ihr stehen verschiedenste Informationen zu verwendetem Browser, Betriebssystem etc.\cite{user-agent-info}.} blocken. Im Gegensatz zu robots.txt sind die Regeln in .htaccess verpflichtend.