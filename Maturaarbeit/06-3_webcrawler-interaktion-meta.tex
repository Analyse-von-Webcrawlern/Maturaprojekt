Der <meta />-Tag ist ein Tag im HTML-Header in welchem verschiedene Zusatzinformationen, wie Autor, Schlagwörter oder ähnliches dem aufrufenden Browser übermittelt werden können. Mit einer dieser Zusatzinformationen des <meta />-Tags kann der Betreiber einer Webseite auch mit den Webcrawlern von Suchmaschinenbetreibern interagieren. Mit dem Tag \emph{<meta name="robots" content="noindex, nofollow" />} kann der Betreiber beispielsweise den Anbietern von Suchmaschinen mitteilen, dass er diese HTML-Seite nicht in die möglichen Suchresultate aufnehmen möchte und der Webcrawler keinen weiteren Hyperlinks auf dieser Seite folgen sollte\cite{meta-w3c}.
Wie bei der Verwendung von robots.txt ist auch diese Einhaltung fakultativ.\cite{meta-beispiel}