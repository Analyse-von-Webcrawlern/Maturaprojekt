\subsection{Motivation}
Heutzutage ist unser alltägliches Leben vom Internet und dessen Erscheinungen stark geprägt. Ob wir nur schnell etwas „googeln“ oder unsere Mails checken und zum wiederholten Male unerwünschte Nachrichten in unserem Posteingang vorfinden, all dies sind das Ergebnis einer Erfindung, welche das Internet revolutionierte. Die Rede ist von Webcrawlern, kleine Programme, welche das Internet nach brauchbaren Informationen durchforsten und diese uns zur Verfügung stellen. 
Laut einer von Imperva Incapsula durchgeführten Untersuchung im Jahre 2016 stammen lediglich 42,8\% \cite{bot-report-2016} des Internet traffics\footnote{Als \emph{Internet traffic} bezeichnet man den Datenfluss im Internet, welcher durch das Aufrufen von Webseiten, Herunterladen von Medieninhalten oder ähnlichem entsteht.} von Menschen. Die restlichen 57,2\% \cite{bot-report-2016} stammen von Webcrawlern und Webspidern, sprich Bots. Webcrawler, oder Bots im Allgemeinen, spielen somit eine wesentliche Rolle in unserem täglichen Onlineleben. Ihre Verhaltensmuster und Absichten zu erkennen und zu verstehen ist somit zu einem wichtigen Gebiet in der informationstechnischen Forschung geworden.