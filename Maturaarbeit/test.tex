\documentclass{scrartcl}

% input encoding
\usepackage[utf8]{inputenc}

% new german spelling
\usepackage[ngerman]{babel}

% choose font
\usepackage[T1]{fontenc}
\usepackage{lmodern}

% KOMA-Script options
\KOMAoptions{%
  parskip=full,%
  fontsize=12pt,
  DIV=calc}

% color and images
\usepackage{xcolor}
\usepackage{graphicx}

% quotes
\usepackage[german=guillemets]{csquotes}

% math
\usepackage{amsmath}

% set special behaviour for hyperlinks in pdfs
\usepackage[breaklinks=true]{hyperref}

\begin{document}
  \title{Verschiedene Blindtexte}
  \author{Malte Schmitz}
  \date{Winter 2015}
  \maketitle

  \tableofcontents

  \section{Gummibärchen}

  Freilebende Gummibärchen \emph{gibt es nicht}. Man kauft sie in \textbf{Packungen} an der Kinokasse. Dieser Kauf ist der \textsc{Beginn} einer fast erotischen und sehr \textcolor{orange}{ambivalenten Beziehung} Gummibärchen-Mensch. Zuerst geniesst man. Dieser Genuss umfasst \texttt{alle Sinne}. Man wühlt in den Gummibärchen, man fühlt sie. Gummibärchen haben eine Konsistenz wie weichgekochter Radiergummi. Die Tastempfindung geht auch ins Sexuelle.\footnote{Das bedeutet nicht unbedingt, dass das Verhältnis zum Gummibärchen ein geschlechtliches wäre, denn prinzipiell sind diese geschlechtsneutral.}

  \begin{quote}
    Im Gegensatz zu anderen Textverarbeitungsprogrammen, die nach dem What-you-see-is-what-you-get-Prinzip funktionieren, arbeitet der Autor mit Textdateien, in denen er innerhalb eines Textes anders zu formatierende Passagen oder Überschriften mit Befehlen textuell auszeichnet.\footnote{Wikipedia über LaTeX}
  \end{quote}

  Nun sind Gummibärchen weder wabbelig noch zäh; sie stehen genau an der Grenze. Auch das macht sie spannend. Gummibärchen sind auf eine aufreizende Art weich. Und da sie weich sind, kann man sie auch ziehen. Ich mache das sehr gerne. Ich sitze im dunklen Kino und ziehe meine Gummibärchen in die Länge, ganz ganz langsam. Man will sie nicht kaputtmachen, und dann siegt doch die Neugier, wieviel Zug so ein Bärchen aushält. (Vorstellbar sind u.a. Gummibärchen-Expander für Kinder und Genesende).

  \subsection{Forscherdrang}

  Forscherdrang und gleichzeitig das Böse im Menschen erreichen den Climax, wenn sich die Mitte des gezerrten Bärchens von Millionen Mikrorissen weiss färbt und gleich darauf das zweigeteilte Stück auf die Finger zurückschnappt. Man hat ein Gefühl der Macht über das hilflose, nette Gummibärchen. Und wie man damit umgeht: \enquote{Mensch erkenne dich selbst!} Jetzt ist es so, dass Gummibärchen ja nicht gleich Gummibärchen ist. Ich bevorzuge das klassische Gummibärchen, künstlich gefärbt und aromatisiert. Mag sein, daß es eine Sentimentalität ist. Jedenfalls halte ich nichts von neuartigen Alternativ-Gummibärchen ohne Farbstoff (\enquote{Mütter, mit viel Vitamin C}), und auch unter den konventionellen tummeln sich schwarze Schafe: die schwarzen Lakritz-Bärchen.

  Wenn ich mit Xao im Kino bin, red ich ihm so lange ein, daß das die besten sind, bis er sie alle isst. Sie schmecken scheusslich und fühlen sich scheusslich an. Dagegen das schöne, herkömmliche Gummibärchen: allein wie es neonhaft vom Leinwandleuchten illuminiert, aber ganz ohne die Kühle der Reklameröhren! Die nächste prickelnde Unternehmung ist das Kauen des Gummibärchens. Es ist ein \enquote{Katz-und-Maus-Spiel}. Man könnte zubeissen, lässt aber die Spannung noch steigen. Man quetscht das nasse Gummibärchen zwischen Zunge und Gaumen und glibscht es durch den Mund. Nach einer Zeit beisse ich zu, oft bei nervigen Filmszenen. Es ist eine animalische Lust dabei. Was das schmecken angeht, wirken Gummibärchen in ihrer massiven Fruchtigkeit sehr dominierend. Zigaretten auf Gummibärchen schmecken nicht gut.

  \subsection{Etwas Mathematik}

  Der Satz des Pythagoras lässt sich folgendermaßen formulieren: Sind $a$, $b$ und $c$ die Seitenlängen eines rechtwinkligen Dreiecks, wobei $a$ und $b$ die Längen der Katheten und $c$ die Länge der Hypotenuse ist, so gilt $a^2 + b^2 = c^2$.

  Darüber hinaus gilt $\alpha^{22} + \beta_{12} = \gamma^2_a$. Anschließend wenden wir
  \[ \sum^n_{i=1}i = \frac{n(n+1)}2 \]
  an, was auch als Summenformel bekannt ist.

  Wenn man mehrere Zeile hat, kann man diese auch ausrichten. Dabei gilt
  \begin{align*}
    \sqrt{x^4} &= x^2\\
    \lim_{n\to\infty} \frac 1{n^2} &= 0 \\
    \int_{-1}^2 x\, \mathrm{d}x &= \left[ \frac12 x^2 \right]_1^2
  \end{align*}

  \subsection{Ergänzende Anmerkungen}

  Anführen sollte man auch noch: 
  \begin{enumerate}
    \item Manche mögen die Grünen am liebsten,
    \item manche die Gelben.
    \item Ich mag am liebsten die Roten.
    \begin{enumerate}
      \item Sie glühen richtig rot,
      \item und ihr Himbeergeschmack fährt wie Napalm über die Geschmacksknospen.
    \end{enumerate}
  \end{enumerate}
  Eine meiner Lieblingsphantasien, wo es um Gummibärchen geht, ist der Gummibär. Ich will einen riesigen Gummibären. Jeder wahre Gummibärchen-Gourmet wird mich verstehen. Ebenfall phantasieanregend können sie eingesetzt werden zum Aufbau verschiedener Orgiengruppen-Modelle oder als Demonstrationsobjekt für wirbellose Tiere. Abgesehen vom diabolischen Lustgewinn müsste man die Bärchen gar nicht zerreissen. Sie sind ja durchscheinend. Zu behaupten, dass sich im Gummibärchen das Wesen aller Dinge offenbart, finde ich keinesfalls als gewagt. Wer schon einmal über einem roten Gummibärchen meditiert hat, weiss von diesen Einsichten.

  \subsection[Im Kino]{Am Ende landet das ganze wie immer im Kino}

  Wenn ich das Kino verlasse oder die Packung einfach leergegessen ist, habe ich meist ein Gefühl, als hätte mir einer in den Magen getreten.
  \begin{itemize}
    \item Hier schläft die gesteigerte Intensität -- als deren Ursache den Gummibärchen durchaus der Charakter einer Droge zuerkannt werden kann -- ins Negative um, in den überdruss.
    \item In dichter und geraffter Form spiegelt sich im Verhältnis zum Gummibärchen eine menschliche Love-Affair wider.
    \item Nie wieder Gummibärchen, denke ich jedesmal. In der Zwischenzeit lächle ich dann über den Absolutheitsanspruch den diese Momente erheben. Schon zu Hause beunruhigen mich wieder Gerüchte über einen Marktvorstoss der Japaner mit Gummireis oder Gummischweinen.
    \item Und wieder und wieder geht es mir durch den Kopf: Gummibärchen sind Spitze.
  \end{itemize}

  \section{Der Blindtext-Fall}

  Sie erinnern sich. Der Blindtext-Fall im vorigen Jahr. Nun will Karl noch nach Canossa. Und Claudia heiratet zur Busse Copperfield. Jeden Morgen entzünden sie eine Kerze. Jeden Nachmittag ist eine Runde Rosenkranz fällig. Zur Heiligen Marie. Weil Karl mit dem Zopf der Claudia mit dem Smile optisch nette Koran-Typo aufs Mieder hat sticken lassen. Heiliger Blindtext am Busen.

  Bumm. Da lässt der Mullah nicht mit sich scherzen. Blindtext killt Chanel, Islam erklärt Karl den Krieg, das Abendland zittert. Der Blindtext-Fall ist geboren. Die Geschichte des Blindtextes und seiner Texter wird aufgeblättert. Endlich. Was wissen Sie über Blindtext? Katholischen nimmt man für Kochbücher, evangelischen für Bauhausmöbelprospekte, hebräischer wird in Hollywood verfilmt, atheistischer ist für Procter \& Gamble Waschmittel, arabischer ist nicht.

  \begin{table}
    \centering
    \begin{tabular}{l|lr}
      \textbf{Jahr} & \textbf{Prozessor} & \textbf{MHz} \\ \hline
      1975 & 6502 (C64) & 1 \\
      1985 & 80386 & 16 \\
      2005 & Pentium 4 & 2\,800 \\
      2030 & Phoenix 3 & 7\,320\,000
    \end{tabular}
    \caption{Und weiter? Zu wem beten Karl und Claudia jeden Tag als Busse für ihre Blindtext-Sünde? Zu ihr. Zur Heiligen Marie Antoinette. Madame ging schön aufs Schafott.}
    \label{tab:prozessoren}
  \end{table}

  Welch eine Haltung in \autoref{tab:prozessoren} dargestellt wird. Sie weiss, sie kriegt den Kopf ab. Aber vorher pudert sie ihn noch, beisst sich auf die Lippen von wegen Lippenrot, kneift sich in die Wangen von wegen Wangenrot. Und sie weiss, sie wird den Kopf verlieren. Oben ab. Und es stört die Marie nicht. Diese Haltung verehren die Blindtexter. Du weisst, du wirst gecuttet. Aber du gibst alles. Sainte Marie, steh uns bei. The english call it the holy attitude of SM.

  Des Blindtexters Heiliges Tier ist das Schwein. Es atmet und furzt, frisst und säuft, um verwurstet zu werden. Wie ähnlich doch dem Blindtext, der nur entsteht, um zerlegt zu werden. Was sagt der Art Director zu Faust? ... denn alles, was entsteht, ist wert, dass es zugrunde geht ...

  Lohnt es sich nun zu clustern, was beim Art Direktor hinten rauskommt? überlassen wir das den Metzgern. So sprechen sie vom Blutwurst-Blindtext, wenn grobe Originalblindtextbrocken in einer ansonsten undefinierbaren Blindtextmasse zu finden sind. Sie sprechen vom Schinken-Blindtext, wenn ein runder geschlossener Blindtext am Stück an einem Foto-Knochen montiert ist. Und es gibt das Blindtext-Filetstück. Das ist der seltene Fall, dass ein Blindtext vom Art Direktor so genommen wird, wie er ist. Psychologen der UCLA, der University of California Los Angeles, haben im Mai dieses Jahres herausgefunden, dass es Art Direktoren gibt, die aufgrund von Geschlecht, Hautfarbe, Religionszugehörigkeit oder Regionalität bisweilen nicht anders können als.

  \begin{figure}
    \centering
    \includegraphics[width=6cm]{flower}
    \caption{Ein Beispiel: Ein männlicher Art Direktor aus Mönchengladbach kann folgendes Stück Blindtext nicht zerhacken: Borussia ist die beste Fussballmannschaft von allen. Sie ist einfach viel besser als die Schweine-Bayern. Sie spielt elegant und intelligent, frisch und mitreissend. Und Effenberg ist der King.}
    \label{fig:flower}
  \end{figure}

  Der AD aus MG nimmt diesen Blindtext als Filet. Der AD aus München macht daraus Blindtext-Gulasch, wenn nicht -Tatar. Die Geschichte des Blindtextes und seiner Texter ist von Natur aus blutig. Die Blindtexter sind die Heiligen. Ihre Werke werden gemartert. Alle Formen der gepflegten Folter finden Anwendung: Kopf ab, unten ab, rechts ab, links ab, vierteilen, stückeln, in Blöcke hacken, dehnen. Die Art Direktoren sind die Schlächter. Warte, warte nur ein Weilchen, dann kommt Hamann auch zu dir. Wurde gerade mit Schimanski als Art Direktor verfilmt. Pfeifen ADs nicht ständig den Hamann-Song? Wofür steht eigentlich AD? Ist es nicht die MTV-Schreibweise für das schwäbische Tschüs? Die Geschichte des Blindtextes und seiner Texter ist so alt wie die Menschheit, und noch nie konnte sie ganz erzählt werden. Denn irgendwann kommt immer der AD.

  \section{Weit hinter den Wortbergen}

  Weit hinten, hinter den Wortbergen, fern der Länder Vokalien und Konsonantien leben die Blindtexte. Abgeschieden wohnen Sie in Buchstabhausen an der Küste des Semantik, eines grossen Sprachozeans. Ein kleines Bächlein namens Duden fliesst durch ihren Ort und versorgt sie mit den nötigen Regelialien.

  Es ist ein paradiesmatisches Land, in dem einem gebratene Satzteile in den Mund fliegen. Nicht einmal von der allmächtigen Interpunktion werden die Blindtexte beherrscht -- ein geradezu unorthographisches Leben.

  Eines Tages aber beschloss eine kleine Zeile Blindtext, ihr Name war Lorem Ipsum, hinaus zu gehen in die weite Grammatik. Der grosse Oxmox riet ihr davon ab, da es dort wimmele von bösen Kommata, wilden Fragezeichen und hinterhältigen Semikoli, doch das Blindtextchen liess sich nicht beirren. Es packte seine sieben Versalien, schob sich sein Initial in den Gürtel und machte sich auf den Weg.

  \begin{description}
    \item[Das Schlagwort] steht am Anfang einer Zeile und wird hervorgehoben, während der zugehörige
    \item[Text] dahinter in normaler Schrift erscheint.
  \end{description}

  Als es die ersten Hügel des Kursivgebirges erklommen hatte, warf es einen letzten Blick zurück auf die Skyline seiner Heimatstadt Buchstabhausen, die Headline von Alphabetdorf und die Subline seiner eigenen Strasse, der Zeilengasse. Wehmütig lief ihm eine rethorische Frage über die Wange, dann setzte es seinen Weg fort.

  Unterwegs traf es eine Copy. Die Copy warnte das Blindtextchen, da, wo sie herkäme wäre sie zigmal umgeschrieben worden und alles, was von ihrem Ursprung noch übrig wäre, sei das Wort und und das Blindtextchen solle umkehren und wieder in sein eigenes, sicheres Land zurückkehren.

  Doch alles Gutzureden konnte es nicht überzeugen und so dauerte es nicht lange, bis ihm ein paar heimtückische Werbetexter auflauerten, es mit Longe und Parole betrunken machten und es dann in ihre Agentur schleppten, wo sie es für ihre Projekte wieder und wieder missbrauchten.

  Und wenn es nicht umgeschrieben wurde, dann benutzen Sie es immer noch.
\end{document}