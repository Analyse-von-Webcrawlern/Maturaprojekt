\subsubsection{GoogleBot}
Einer der wohl bekanntesten Webcrawler ist der \emph{GoogleBot}. GoogleBot ist der Webcrawler von Google, der größte Suchmaschinenanbieter weltweit mit über 90\% Marktanteil\cite{google-marktanteil}. Google setzt hierbei Algorithmen ein, um die GoogleBots zu steuern und festzulegen, welche Webseite wann und wie oft abgerufen wird\cite{googlebot-googleconsole}. Die gecrawlten Seiten werden dann von Google indexiert\footnote{Die Indexierung einer Webseite durch einen Suchmaschinenbetreiber bedeutet nichts anderes, als dass die Webseite von einem Suchmaschinenbetreiber gefunden wurde. Indexierte Webseiten werden dann in einem zweiten Moment als Vorschlag auf eine Suchanfrage ausgegeben.\cite{was-ist-indexieren}} und können im Bedarfsfall abgerufen und als Suchergebnisse in Bruchteilen einer Sekunde ausgegeben werden.\\
Da GoogleBot der weltweit größte operierende Webcrawler ist, ist er auch einer der am besten dokumentierten Webcrawler. Hier ein paar Fakten über den GoogleBot:
\begin{itemize}
	\item Im Jahre 2012 hatte Google mit seinem GoogleBot bereits 96\%\cite{googlebot-facts-percentage} aller Webseiten weltweit gecrawlt, Tendenz steigend. 
	\item Der User Agent von GoogleBot lautet\\\texttt{Mozilla/5.0 (compatible; Googlebot/2.1;\\+http://www.google.com/bot.html)}\cite{googlebot-user-agent}. 
	\item Im Durchschnitt besuchen GoogleBots eine Webseite 187mal am Tag\cite{googlebot-facts-incapsula}.\footnote{\label{footnote:incapsula-studie}Wert aus einer 2014 von Incapsula (amerikanisches Softwareunternehmen) durchgeführten Studie, in der sie 10.000 Webseiten ihrer Kunden über 30 Tage lang beobachteten.}
	\item Durchschnittlich stammt jeder fünfundzwanzigste Besuch von einem GoogleBot \emph{nicht} von GoogleBot. Dies bedeutet dass bösartige Bots den User Agent vom GoogleBot missbrauchen, um sich als dieser auszugeben und so unerkannt zu bleiben.\cite{googlebot-facts-incapsula}\textsuperscript{\ref{footnote:incapsula-studie}}
	\item Google verwendet mehrere Crawler, welche für unterschiedliche Zwecke entworfen wurden\cite{googlebot-user-agent}.
\end{itemize}
\label{subsub:googlebot}

