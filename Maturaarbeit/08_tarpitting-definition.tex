\subsubsection{Definition}
Unter Tarpitting versteht man den Vorgang, bei welchem ein Webcrawler mithilfe einer Tarpit durch verschiedenste Maßnahmen angelockt und später ausgebremst bzw. ausgeschaltet wird. Tarpits können dabei auf allen Layern des ISO/OSI-Modells\footnote{Das ISO/OSI-Modell ist ein Schichtenmodell, welches heutzutage das theoretische Modell für die Datenkommunikation zwischen Computern in einem Netzwerk und im Internet bildet\cite[430]{gallenbacher-iso-osi}.} implementiert werden\cite{tarpit-wikipedia}. Das Prinzip einer Tarpit ist dabei analog zu ihrem Namensgeber, einer Teergrube (engl. Tarpit), in welcher beispielsweise Tiere steckenbleiben. Tarpits arbeiten heutzutage oft in Kombination mit einem Honeypot\footnote{Ein \emph{Honeybot} ist ein System, welches versucht, Angreifer anzulocken, mit dem Ziel, Informationen wie die IP-Adresse oder den Standpunkt dieser herauszufinden oder neue Angriffsmethoden zu erkennen. Hierbei ist ein Honeypot ein \glqq normales\grqq\space System, welches jedoch bewusst bekannte Sicherheitsschwachstellen und -lücken aufweist\cite{def-honeybot}.}, welcher die Bots anlocken soll. Die Ausdrücke Honeypot und Tarpit werden deshalb heutzutage oft als Synonym verwendet\cite{honeypot-tarpit-synonym}.
Da tarpitting ein Mittel zur Bekämpfung von Bots ist, wird versucht es weitestgehend geheim zuhalten um den Entwicklern von Bots keine möglichen Schwachstellen aufzuzeigen. Deshalb ist, wenn überhaupt, nur eine unvollständige Dokumentation öffentlich einsehbar. Auch die IP-Adressen und URLs von Tarpits werden geheimgehalten.
\label{subsub:tarpitting}