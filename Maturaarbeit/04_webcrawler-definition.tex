\subsubsection{Definition}
In der heutigen Zeit fällt oft der Begriff \emph{Bot}. Seltener werden die Begriffe \emph{Webcrawler} oder gar \emph{Webspider} und \emph{Harvester} verwendet, obwohl diese meistens zutreffender wären. Die oben angeführten Bezeichnungen sollen hier nun nochmals genauer definiert werden: 
\begin{description}
	\item[Bot:] Ein Bot ist ein Programm, welches autonom oder nach Erhalt von Eingaben Aufgaben und Aktionen im Internet ausführt. Bot ist hierbei die Kurzform des Begriffs Ro\underline{bot}\cite{def-bot}.
	
	\item[Webcrawler:] Ein Webcrawler ist ein gängiges Beispiel eines Bots\cite{def-bot}. Hyperlinks\footnote{Ein Hyperlink ist ein Element in einer Webseite (Wort, Text oder auch Bild), mit welchem man mittels eines Mausklicks auf eine andere Webseite gelangt.} folgend, speichert ein Webcrawler den Inhalt von Webseiten, um diese dann in einem zweiten Moment von Suchmaschinenbetreibern mit Algorithmen auswerten und bewerten zu lassen\cite{def-bot}. Webcrawler können jedoch auch für schadhafte Zwecke missbraucht werden und beispielsweise alle E-Mail-Adressen aus einer Webseite herausfiltern. 
	
	\item[Webspider:] Ein Webspider ist ein Bot, welcher sich von Hyperlink zu Hyperlink durch das Internet bewegt und hierbei Webseiten ausfindig macht und die Ergebnisse wiederum meist an einen Suchmaschinenbetreiber meldet\cite{def-spider}. Seine Funktion entspricht somit der eines gängigen Webcrawlers.
	
	\item[Harvester:] Ein Harvester ist eine spezielle Form eines Webcrawlers. Ein Harvester durchforstet dabei gezielt das Internet nach E-Mail-Adressen, um in einem zweiten Moment an diese Adressen Spamnachrichten\footnote{Spamnachrichten sind Nachrichten, welche vom Empfänger nicht erwünscht sind.} verschicken zu können.
\end{description}
Wie aus den oben aufgeführten Definitionen hervorgeht, beschreiben die Begriffe stets ein ähnliches Prinzip. Vor allem die Ausdrücke Webcrawler, Webspider und Harvester sind in ihrer Bedeutung nahezu ident. Deshalb werden diese Begriffe oft als Synonyme verwendet und unter dem Oberbegriff Bot zusammengeführt.
\label{subsub:def}