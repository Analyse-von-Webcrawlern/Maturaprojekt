\subsubsection{Interaktionsmöglichkeiten}
Das Internet ist ein öffentliche Medium und somit sind auch die Webseiten für jedermann zugänglich. Diese Tatsache hat maßgeblich zur Beliebtheit und zum Erfolg des Internets beigetragen und genau deshalb setzen so viele Unternehmen auf das Internet als Werbefläche. Haben Webseitenbetreiber also überhaupt eine Möglichkeit bösartige oder unerwünschte Webcrawler von ihren Webseiten fernzuhalten? In den vergangen Jahren wurden jedenfalls vermehrt Mittel geschaffen, um Betreibern von Webseiten die Möglichkeit zu geben mit Webcrawlern zu interagieren. Auf die bekanntesten dieser Mittel soll nun kurz eingegangen werden.
\begin{description}
	\item[robots.txt]
	Das Robots Exclusion Protocol, besser bekannt als \emph{robots.txt} wurde 1994\cite{robots-date} vom Holländer Martijn Koster\cite{robots-martijn} erfunden und hat sich zum de-facto-Standard in Sachen Interaktion mit Webcrawlern entwickelt, obwohl es nie ein offizieller Standard wurde.\cite{robots-date}\\
Um das Robots Exclusion Protocol zu verwenden, muss eine Datei mit dem Namen \emph{robots.txt} im Wurzelverzeichnis der Webseite abgespeichert werden. In dieser Datei können dann Regeln festgelegt werden, die besagen, welcher Webcrawler welche Dateien und Ordner besuchen darf bzw. welche nicht. Das einhalten dieser Regeln ist jedoch fakultativ, weshalb Webcrawler die Regeln der robots.txt auch einfach ignorieren können.
	\item[.htaccess]
	\emph{.htaccess} ist der Name einer Textdatei, die für die Konfiguration eines Apache Webservers verwendet wird. Da der Webserver von Apache die im Internet mit einem Marktanteil von 46,9\%\cite[Stand 24.03.2018:][]{statistik-webserver} am weitesten verbreitete Serversoftware ist, ist auch die Verbreitung von .htaccess dementsprechend hoch.
%TODO LÖSCHEN
%Die .htaccess-Datei kann dabei an jeden x-beliebigen Ort innerhalb des Webservers platziert werden. Die Regeln sind dann für diesen Ordner und alle Unterordner wirksam\cite{htaccess}.
Apache akzeptiert hierbei eine Vielzahl an Regeln in der .htaccess-Datei; darunter auch Regeln, welche Anfragen anhand der IP-Adresse und/oder des User Agent\footnote{Ein \emph{User Agent} ist eine Zeichenfolge, welche bei jedem Aufruf einer Seite an den Webserver übermittelt wird. In ihr stehen verschiedenste Informationen zu verwendetem Browser, Betriebssystem etc.\cite{user-agent-info}.} blocken. Im Gegensatz zu robots.txt sind die Regeln in .htaccess verpflichtend.
	\item[<meta />]
	Der <meta />-Tag ist ein Tag im HTML-Header in welchem verschiedene Zusatzinformationen, wie Autor, Schlagwörter oder ähnliches dem aufrufenden Browser übermittelt werden können. Mit einer dieser Zusatzinformationen des <meta />-Tags kann der Betreiber einer Webseite auch mit den Webcrawlern von Suchmaschinenbetreibern interagieren. Mit dem Tag \emph{<meta name="robots" content="noindex, nofollow" />} kann der Betreiber beispielsweise den Anbietern von Suchmaschinen mitteilen, dass er diese HTML-Seite nicht in die möglichen Suchresultate aufnehmen möchte und der Webcrawler keinen weiteren Hyperlinks auf dieser Seite folgen sollte\cite{meta-w3c}.
Wie bei der Verwendung von robots.txt ist auch diese Einhaltung fakultativ.\cite{meta-beispiel}
	\item[Automated Content Access Protocol]
	Das \underline{A}utomated \underline{C}ontent \underline{A}ccess \underline{P}rotocol (kurz ACAP) ist ein 2007 gegründetes Projekt, welches die Copyright-Ansprüche von Inhaltserstellern im Internet besser schützen soll.\cite{acap-was} Es will hierbei das weitverbreitete robots.txt nicht verdrängen, sondern eine Alternative bieten. ACAP wird heutzutage kaum verwendet,  da es namhafte Internetkonzerne, wie etwa Google, nicht nutzen\cite{acap-google}.
	%TODO: löschen wenns nimma braucht
	%\item[Webserver spezifische Mittel]
	%Abhängig von der gewählten Webserversoftware sind verschiedene Plugins, Module,  Erweiterungen, Optionen, usw. verfügbar mit denen sich der Zugriff regulieren lässt. Bwwährt hat sich die Devise: Je bekannter die Software, desto  mehr Plugins gibt es.
So ist das Modul \emph{mod\textunderscore authz\textunderscore host} beispielsweise eines der beliebtesten Module wenn es um Anfragebeschränkungen für den Apache Server geht\footcite{apache-module-beschraenkung}.
\end{description}
