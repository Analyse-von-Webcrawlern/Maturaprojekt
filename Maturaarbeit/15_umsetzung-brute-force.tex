\subsubsection{Brute-Force-Tarpit}
Eine der gängigsten  Arten um Passwörter zu knacken ist die sogenannte \emph{Brute-Force}-Methode, zu Deutsch \glqq rohe Gewalt\grqq-Methode. Hierbei werden gängige Kombinationen aus Benutzername und Passwort in einem Anmeldeformular ausprobiert, mit der Hoffnung, dass eine der Kombinationen richtig ist.\\
Die erstellte Tarpit bietet auch eine Anlaufstelle für solche Webcrawler. Unter der URL \url{http://maturaprojekt.ddns.net/wp-login.php} ist ein WordPress-Login simuliert. WordPress ist mit 60\% Marktanteil\cite{wordpress-verbreitung} das am weiten Verbreiteste Content-Management-System\footnote{Ein Content-Management-System (CMS) ist eine Software, welche einen oder mehrere Anwender bei der Erstellung, Barbeitung und Verwaltung von Inhalten, meist Webseiten, unterstützt.} und erfreut sich großer Beliebtheit. Durch die einfache und schnelle Installation bzw. Einrichtung ist Wordpress auch bei Anfängern eine beliebte Software für das Erstellen einer Webseite. Um Änderungen an der Webseite vornehmen zu können, müssen sich die Anwender über eine Log-In-Form anmelden, welche standardmäßig über \emph{[DOMAINE]/wp-login.php} erreichbar ist. Durch die weite Verbreitung und den allgemein bekannten Standartpfad ist diese Log-In-Form ein häufiges Ziel von Brute-Force-Angriffen. Webcrawler überprüfen, ob sie die zuvor erwähnte Log-In-Form unter der gewohnten Adresse finden und starten dann mit einem Brute-Force-Angriff auf diese Log-In-Maske.\\
VEVETA besitzt ebenfalls eine WordPress-Log-In-Form. Die Struktur dieser stammt aus einer öffentlich zugänglichen Demo von \emph{Softaculous}\footnote{Die Demo ist Einsehbar unter der Adresse \url{https://demos1.softaculous.com/WordPress/wp-login.php}.}, welche so modifiziert wurde, dass sie den Anschein einer gängigen WordPress-Log-In-Form der Webseite \emph{maturaprojekt.ddns.net} erweckt. Die darin enthaltene Eingabeform ist so modifiziert, dass nach dem Absenden der Eingabedaten diese in einem Log zusammen mit IP-Adresse und Datum gespeichert werden. Der hier aufgezeigte Codeabschnitt ist für dies verantwortlich:
\lstinputlisting[language=PHP, firstline=21, lastline=31]{sourcefiles/wp-login.php}
Nach dem Abschicken der Log-In-Daten wird der Aufrufer wieder auf die Seite \emph{/wp-login.php} zurück geleitet, dies erzeugt den Eindruck eines falsch eingegebenen Passworts und der Aufrufer kann seine nächste Kombination aus Benutzername und Passwort ausprobieren. Diese Art der Tarpit dient somit in erster Linie der Analyse von Webcrawlern, welche die Absicht haben, Brute-Force-Angriffe auf einer Webseite auszuüben.
\label{subsub:brufe-force-tarpit}