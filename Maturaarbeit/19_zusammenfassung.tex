Das Internet als ein immer stärker wachsendes Medium beinhaltet bereits heutzutage eine endlos scheinende Menge an Informationen. Menschen sind mit dieser Fülle an Informationen schon seit langem überfordert. Es wurden Mittel geschaffen diese Menge an Informationen für uns Menschen aufzubereiten, nämlich Webcrawler. Sie sind, auch wenn wir sie kaum bemerken, zu einem der wichtigsten Hilfsmittel in unserer heutigen Onlinewelt geworden. Doch wie so vieles haben auch sie Schattenseiten: Einige Webcrawler versuchen Inhalte von Webseiten zu kopieren, Passwörter zu knacken oder E-Mail-Adressen für einen späteren Versand von Spamnachrichten zu sammeln.\\
Durch einen aggressiven Aufruf von Webseiten versuchen bösartige Webcrawler so schnell als möglich an eine Fülle an Informationen zu gelangen und nehmen dabei auch einen Serverabsturz in Kauf. Regeln der robots.txt werden hierbei bewusst verletzt, ein Serveradministrator ist in solchen Fällen oft machtlos.\\
Das Versenden von Spam- oder Scramnachrichten scheint immer noch ein lukratives Business zu sein. Öffentlich einsehbare E-Mail-Adressen werden gestohlen und landen in den Listen von Harvestern. Es finden sich in den Postfächern der Opfer immer wieder fragwürdige aber echt erscheinende Mails. Diese Mails gehören ohne zu zögern in den Papierkorb, ein Öffnen dieser könnte bereits das Ausführen eines Virus starten. Das Imitieren einer E-Mail von einem echten Unternehmen bedeutet auch für dieses einen großen Schaden. Die sinkenden Reputation nach so einem Vorfall ist für ein weltweit operierenden Unternehmen meist der größte finanzielle Schaden, den sie sich vorstellen können.\\
Die in den Mails enthaltenen Viren können unter Umständen auch das betroffene Endgerät übernehmen und in ein Botnetz einhängen. Dieses kann dann mit einem Klick aktiviert werden und von praktisch jedem Land der Welt aus starten Endgeräte DDoS-Attacken oder Brute-Force-Versuche auf Loginformen von populären System, wie etwa Wordpress. Listen von beliebten Kombinationen aus Benutzernamen und Passwörtern werden hier meist über Tage hinweg ausprobiert. Das Opfer ist diesen Angriffen schutzlos ausgeliefert, da die vermeintlichen Angreifer normale Endgeräte sind, welche von unwissenden Nutzern wie Du und Ich verwendet werden und man sie deshalb nicht aussperren kann. Das Opfer kann hierbei nur hoffen, dass seine Kombination aus Benutzernamen und Passwort nicht erraten wird, bei einem starken Passwort ist dass jedoch sehr wahrscheinlich.