Ziel dieses Projektes ist, die vorher eingeführten Konzepte der Webcrawler mithilfe verschiedenster Arten von HTTP-Tarpits zu beweisen. Hierbei werden, wie unter Punkt \ref{umsetzung_grundlegendes} beschrieben, drei verschiedene Formen einer HTTP-Tarpit umgesetzt. Webcrawler sollen dann in diese Tarpit geraten. Ihr Verhalten innerhalb dieser Tarpit wird dann mitgeloggt und anschließend mithilfe von Drittanbietersoftware auf crawlertypische Verhaltensmuster hin analysiert. Des Weiteren wird der Zusammenhang zwischen Webcrawlern und allgemein bekannten Gefahren bei der Benutzung des Internets, wie beispielsweise das Veröffentlichen der E-Mail-Adresse auf einer Webseite oder das Benutzen von schwachen Passwörtern, untersucht und die Risiken und Auswirkungen dieser Gefahren aufgezeigt.