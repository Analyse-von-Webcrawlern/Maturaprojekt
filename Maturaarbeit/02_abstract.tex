\addsec{Abstract}
\begin{abstract}
	Webcrawler sind in der heutigen Zeit nicht mehr wegzudenken. Sie durchforschen und analysieren für große Internetdienstanbieter wie Google oder Amazon das Internet und liefern ihnen brauchbare Informationen. Einige von ihnen haben jedoch auch das Ziel, E-Mail-Adressen aus Webseiten zu filtern, Passwörter zu hacken und Schaden anzurichten. Diese Arbeit hat sich das Ziel gesetzt, Webcrawler mithilfe einer im Internet weit verbreiteten Technologie, der Tarpit, zu fangen und ihr Verhalten zu dokumentieren. Dieses Verhalten wird auf crawlertypische Muster, wie die Suche nach E-Mail-Adressen oder das Indexieren von Webseiten, hin untersucht und offenbart somit die Absichten der zuvor gefangenen Webcrawler. Der erste Teil dieser Arbeit beschäftigt sich hierbei mit den theoretischen Aspekten eines Webcrawlers. Hier werden anhand von einschlägigen Beispielen die Funktionsweise von Webcrawlern beleuchtet und ihre Aufgaben, welchen sie im Internet nachgehen, aufgezeigt. Des Weiteren wird auch auf die Praxis des Tarpittings näher eingegangen und ihre Arbeitsweise dem Leser nähergeführt. Im zweiten Teil dieser Arbeit werden die zuvor erläuterten Konzepte in einem Experiment umgesetzt, in welchem eine HTTP-Tarpit realisiert wird. Mithilfe dieser Tarpit soll die Existenz von Webcrawlern bewiesen werden, indem sie auf einer speziell für diese Aufgabe präparierten Webseite Hyperlinks aufrufen, welche auf nicht existierende Ziele verweisen. Anstelle eines 404-Errorcodes meldet der Webserver dann eine zufällig generierte Webseite, welche weitere Hyperlinks enthält. So wird versucht die Webcrawler für eine möglichst große Zeitspanne auf dem System zu halten, damit ihr Verhalten anhand von den generierten Logfiles analysiert werden kann. Hierbei geht diese Arbeit auch auf den Aufbau der zum Experiment gehörenden Webelemente, sprich den Webserver und die dazugehörigen Webseiten, ein. Abschließend werden die Logfiles ausgewertet, ein Resümee gezogen und die Ergebnisse der Tarpit präsentiert.	
\end{abstract}
