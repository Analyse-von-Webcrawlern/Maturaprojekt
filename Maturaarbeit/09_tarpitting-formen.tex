\subsubsection{Formen}
Wie bereits erwähnt, können Tarpits theoretisch in jedem Layer des ISO/OSI-Modells implementiert werden. Doch vor allem die Implementation und der Betrieb auf den unteren Schichten des ISO/OSI-Modells ist aufwendig, mühsam und fehleranfällig, da es dort keine Mittel zur Identifikation der Pakete und Einteilung derselbigen in \glqq Gut\grqq\space und \glqq Böse\grqq\space gibt. Deshalb werden heutzutage Tarpits hauptsächlich im IP-, TCP- und Application-Layer eingesetzt.\\
Während Tarpits auf dem IP-Layer versuchen durch minimieren der versendeten Daten die Verbindung künstlich in die Länge zu ziehen, nutzen TCP-Tarpits den Three-Way-Handshake\footnote{Der Three-Way-Handshake wird vom Transmission Control Protocol (TCP) verwendet um eine Verbindung aufzubauen.} aus, um potentielle Angreifer in die Irre zu führen. Tarpits auf dem Application-Layer, lassen sich in zwei große Gruppen unterteilen: SMTP-Tarpits und HTTP-Tarpits. SMTP-Tarpits behindern Mailserver, welche unerwünschte E-Mails, vielfach als Spam bezeichnet, versenden, in dem sie beispielsweise den Verbindungsaufbau, welcher vor dem versenden stattfinden muss, verlangsamen\cite{tarpit-smtp}. Die Funktionsweise von HTTP-Tarpits wird im folgenden Abschnitt \ref{subsub:http-tarpit} genauer beschrieben.
